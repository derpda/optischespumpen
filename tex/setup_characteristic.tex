\section{Experimental set-up}
\begin{figure}[h]
\centering
\includegraphics[width=1.0\linewidth]{graphics/generalsetup}
\caption[Basic experimental set-up]{Basic set-up of the experiment. For the various tasks, parts can be placed onto the optical bench. The glass cell can be removed from the central unit. \cite{anleitung} (modified)}
\label{fig:general setup}
\end{figure}
At the core of the experimental set-up is the glass cell with the Rubidium vapor and the buffer gas. It can be taken in and out of the central unit, which in turn houses four sets of Helmholtz coils. Another such pair is directly attached to the casing of the glass cell, along with a radio frequency generator and an appropriate frequency measuring device.\\

\textbf{A laser diode} provides coherent light in the energy range needed to pump the desired hyperfine state of the Rubidium atoms in the glass cell. Laser diodes send out linearly polarized light with a small spectral width. Frequency and intensity vary with the temperature of the diode as well as the current running through it, which is why the diode is kept at constant temperature using a Peltier element. For the diode to start emitting light, a certain current threshold has to be reached. From then on, the intensity depends linearly on the current if temperature is kept constant. However, mode jumps at certain currents disturb the linearity. Mode jumps occur when the number of standing waves in the resonator changes. Measurements need to be taken in areas that do not include such jumps.\\

The beam is collimated by a lens before passing through other optical elements and, after passing through the central unit, is refocused onto a photo-diode. This can be seen in figure \ref{fig:general setup}. The output of the diode is amplified and can then be observed on an oscilloscope, which in turn can be read out by a computer to produce analyzable data.\\ 

The set-up varies greatly from one part of the experiment to another and will thus be explained in detail in the appropriate sections.


\section{Characterization of the laser diode}
For later measurements, it is important to determine the range of supply current in which the diode intensity increases linearly without mode jumps occurring. The gas cell is taken out of the central unit for this part of the experiment. \\

After turning on the peltier element, a few minutes should pass before measurements are started to allow the diode to thermalize. Measurements were taken at $T=\unit{34.3}{\degree}$.\\
Since the photo diode saturated for laser diode currents upwards of $I_L=\unit{65}{mA}$, a neutral filter (D2,6) is used to limit the intensity so that the photo diode barely does not reach saturation.

The intensity of the diode is now measured at supply currents between $\unit{0-90}{mA}$. The results can be seen in figure \ref{fig:characterization}. The threshold current is roughly $\unit{51.6}{mA}$, followed by the linear domain until mode jumps occur at around $\unit{72}{mA}$ to $\unit{82}{mA}$. 
\begin{figure}[htb]
\centering
\includegraphics[width=1.0\linewidth]{graphics/characterization}
\caption[Characteristic curve of the laser diode]{The characteristic curve of the laser diode for supply currents up to $\unit{90}{mA}$. A mode jump can clearly be seen in the uper right corner.}
\label{fig:characterization}
\end{figure}





